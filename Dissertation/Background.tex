%% ----------------------------------------------------------------
%% Background.tex
%% ---------------------------------------------------------------- 

\chapter{Background Research} \label{Chapter:Background Research}

This chapter will summarize the background research of this project, which focuses on the factors influencing international higher education destination choices and the application of open data and data visualisation in higher education. 

\section{Factors Influencing Higher Education Destination Choices
}

The international higher education destination choice is typically explained using “push-pull” framework, a universally accepted framework proposed by McMahon \cite{mcmahon1992higher} in 1992. According to this framework, the decision-making process for the choice of studying abroad has three stages. In the first stage, students are influenced by “push” factors within their home countries and decide to study overseas for their higher education. In the second stage, “pull” factors within countries start to influence international students’ decision and prompt them to choose one country as their destination countries. In the final stage, students are influenced by “pull” factors within universities and decide which destination universities they attend. 

Though the impacts of social events may alter international students’ opinions on higher education destination choices, we can still summarize the most significant “pull” factors that influence the international higher education destination choice from the literature \cite{mazzarol2002push, maringe2007international,petruzzellis2010educational, gong2015chinese} 

“Reputation” is one of the key “pull” factors, which is usually reflected by university rankings. International students prefer to consider countries that have more high ranking universities as their destination countries and universities that have the higher ranking as their destination countries. Hence, it is easy to understand why many international students prefer studying in the USA or the UK for their higher education \cite{mazzarol2002push}.  

“Safety” is another significant factor that influences international students’ decisions. It was identified and categorised into “social cost” in the study of Mozzarol \cite{mazzarol2002push}. The importance of “safety” was also emphasised in Gong’ study \cite{gong2015chinese} where Chinese students take safety seriously in their decision-making process. 

“Cost” is a frequently mentioned “pull” factor, which consists of tuition fees, living fees and travel expenses. This situation was especially true for students from India, China, and Indonesia, so students in these countries were more likely to consider the costs as an important factor \cite{gong2015chinese}.

“Word-of-mouth” is a crucial factor that should not be neglected, because several decision makers, like parents of international students, are often involved in the decision-making process for the choice of studying abroad. Mozzarol \cite{mazzarol2002push} suggested in his study that parental influence was particularly strong when undergraduate students choose a destination country or university. 

Moreover, “environment” is a key factor in the international higher education destination choice. It can be measured in several scopes, like learning environment. Mozzarol \cite{mazzarol2002push} found the climate and the lifestyle could greatly influence the attractiveness of a destination country or university. For example, students in South-east Asia had a preference in Australia due to its warmer climate \cite{mazzarol2002push}.

Apart from the factors above, there are still some factors, such as geographic proximity and social links that influence the international higher education destination choice, but the significance of those factors varies from different international student populations. For example, Gong \cite{gong2015chinese} found geographic proximity from China to the destination country was a less important factor in Chinese students’ decision-making process when compared to other factors. On the contrary, Mozzarol \cite{mazzarol2002push} found this factor resulted in the strong flow of Canadian students to the USA, Korean students to Japan and Indonesian students to Australia.

As has been summarized, “reputation”, “safety”, “cost”, “word-of-mouth” and “environment” were the most vital “pull” factors in international students’ decision-making process for the choice of studying overseas. Although the literature \cite{mazzarol2002push, maringe2007international,petruzzellis2010educational, gong2015chinese}  analysed the “pull” factors of destination countries and universities, they are not helpful for those international students who already have decided their destination countries and want to compare “pull” factors of different cities. Moreover, economic and social events are likely to change the “pull” factors in the international higher education destination choices. In this way, it is meaningful to analyse and summarise current key factors influencing international students’ decision-making process for destination cities and universities.


\section{Open Data in Higher Education 
}

Open data is data that anyone can access, use or share \cite{Open_Data_Institute}. According to the report from McKinsey \cite{McKinsey}, there are five fields that can take advantages of open data in higher education: improved instruction and personalized lessons, matching students to universities or programs, matching students to employment, transparent education financing, and more efficient university governance. Therefore, we will discuss the benefits and applications of open data in higher education from these five aspects.

\textbf{Improved instruction}: Universities or teachers can improve the instruction and personalize lessons by analysing open data on student performance and learning styles. The popularity of MOOCs (Massive Online Open Courses), for example, attracts a great number of students to learn these online courses. The performance and learning styles of the students who attend the MOOCs can help universities or teachers to improve the quality of courses and make their online courses better. 

\textbf{Matching students to universities or programs}: As discussed in section 2.1.1, the international students’ decision-making process is influenced by several key factors, so analysing open data about these factors can help prospective students find suitable universities or programs. For example, Crime Map developed by police.uk can provide open data about criminality in a particular region, and it is helpful for students who care about safety in their decision-making process for destination universities in the UK.

\textbf{Matching students to employment}: Open data has a potential benefit in matching the skills students possess with the skills that employers need. Companies can analyse open data about universities and adjust their recruitments strategies in different universities. Similarly, universities can adjust the syllabus of programs and allow students to learn skills or knowledge that companies need according to the open data about job marketing.

\textbf{Transparent education financing}: One way to control rising higher education costs is to create more informed consumers. Therefore, open data can make a significant contribution to transparent education financing. College Scorecard launched by the White House is an online tool to compare the actual tuition costs at various colleges and different financing options. It can help students who are deciding whether they can afford higher education to find suitable programs.

\textbf{More efficient university governance}: Open data can improve the transparency of higher education governance and providing evidence to inform policy. Besides, open data can stimulate communication and partnership between universities. Equipment.data, for example, is an open database that allows the researchers to search for and locate the equipment within their own or at other institutions, and it can improve sharing of equipment and stimulate cooperation between institutions.

Open data can come from any sources, but the primary source of open data are governments. With the open government initiative, governments have posted a high of number datasets online, such as data.gov.uk in the UK and data.gov in the USA. These open government datasets could not only facilitate government transparency, accountability and public participation, but also contribute to higher education. Apart from governments, organization, universities or even individuals can make their data public. For instance, University of Southampton provides an open data service that allows the public to access some administrative data, and the open data service is employed to develop applications to help students and academic staff. 

The open data formats vary from the platforms and user’s needs. Choosing right formats for open data can assist usability, make management easier and lower the barriers of access \cite{Europeandataportal}. Open data can be categorised into downloadable data and live or feed data. According to the 5-star scheme \cite{berners5star}, downloadable open data has 5 different formats, while most of them are published in CSV format, which is easy-to-understand, highly reusable and machine-readable. On the other hand, sometimes open data is not suitable for download or needs to be updated regularly. Therefore, this kind of open data is normally available in live or feed data. The application programming interface (API) is a kind of live data that makes it easy to efficiently share data and processes. 

Although open data is beneficial for higher education, few benefit can be realized without removing several significant barriers. First of all, the state of open data in higher education is relatively primitive and more datasets should be published and applied in higher education. Secondly, parents, students and teachers are likely to worry about their piracy and be reluctant to open educational data about them. Thirdly, it is hard for people to determine what data should be collected, how to share it and how to use it for higher education. Last but not the least, the significant gaps in technology, funding, and technical capabilities make it difficult for universities or educational institutions to apply open data into their everyday workflows. Therefore, there is still a long way for governments, organizations, universities and individuals to achieve all potential benefits of open data in higher education.



\section{Data Visualisation in Higher Education
}

The effective data visualisation is a crucial tool in the decision-making process. It allows decision makers to analyse large amounts of data quickly, figure out trends and problems efficiently, exchange ideas openly, and influence the decision-making process that eventually lead to success. Therefore, data visualisation is used to help international students’ decision-making process for higher education destination choices.

University ranking is a traditional method for international students to gain insights into the basic information on higher education in different countries or universities. Every year, various university rankings are published by newspapers, magazines or institutions around the world to help international students choose their destination countries or universities for their higher education. However, the problem is that these rankings only include part of factors and ignore other key factors, such as “safety” and “cost” in the international higher educational destination choice. In other words, these rankings cannot offer international students comprehensive descriptions of destination countries or universities. Also, most universities rankings are classic tables or charts, so they cannot allow students to interact with data and information.

Nowadays, various data visualisation techniques are developed by developers around the world. The appearance of these techniques has increased the interactivity and readability of data. Data Driven Documents (D3.js) is a widely-used data visualisation techniques. Specifically, it a flexible and powerful JavaScript library that can be used to add data visualisations on the web pages. By using D3.js, the developers can manipulate DOM objects using data and add many other DOM functions like zoom, click function for any visualisation. D3.js offers flexibility to developers, but it also adds learning cost to developers, especially for non-tech people. Therefore, some easy-to-use data visualisation tools, like Google charts and Highcharts.js, are developed to help developers to utilize some basic charts, such as bar charts and pie charts, to visualize data in a simple way.

Due to the benefits of data visualisation techniques, some of them have been applied in visualising the universities rankings. A web-based application using Google charts and D3.js to visualize the top 400 universities was introduced in \cite{aljehane2015visualizing}. The data employed in this application was based on 2013-2014 Times Higher Education university ranking, which consisted of five different criteria: teaching, research, international outlook, industry income, and citation. This application used a wide range of graphs, like maps, bar charts and pie charts to visualise the location of universities, the score variation of all ranking criteria and the classification of universities by different subjects in each country. Similarly, Bornnman \cite{bornmann2014ranking} introduced a web application that visualized the academic performance of universities based on publications and citations. By using this application, students are able to compare the academic quality or the scientific impact of universities at ease.

Although the applications mentioned above applied data visualisation techniques in university rankings and improved the interactivity and readability of data, the drawback of them is that they were both limited to some particular factors influencing international students’ decision-making process for higher education. As a result, international students cannot obtain sufficient information from these web applications to make decisions for their higher education destination choices. Therefore, open datasets about other key factors should be included and visualized in a web application. However, there are several challenges of combining open data and data visualisation to help international students’ higher education destination choices. The first challenge is that it is possible for there to be no available open data about some key factors in the higher education destination choices. The second one is that some available open datasets might need pre-processing and manipulation before using them. And the third one is what data visualisation techniques should be selected for different open datasets and how to visualize them to help international students’ decision making.



\section{Summary}



This chapter firstly gives an overview of the international students’ destination choices and identifies four key factors influencing their choices. Then, it illustrates the benefits of open data in higher education with some examples and introduces the barriers should be removed to realize those benefits. Afterwards, this chapter introduces some data visualisation techniques and applications in higher education to identify their shortcomings. Lastly, this chapter introduces the challenges of applying open data and data visualisation in higher education destination choices.
