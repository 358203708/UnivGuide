%% ----------------------------------------------------------------
%% Introduction.tex
%% ---------------------------------------------------------------- 

\chapter{Introduction} \label{Chapter:Introduction}
In recent years, there has been a tremendous growth in the number of international students who choose to study abroad for their higher education \cite{ahmad2015investigation}. The decision-making process for higher education destination choices is a highly complicated process that is subject to multiple factors. Understanding how international students’ decision-making process regarding higher education destination choices can help them make informed decisions. Open data is widely used in various areas, and it proves to have great benefits in higher education. On the other hand, well-designed data graphics or charts are usually the simplest and the most powerful way to help the users think visually \cite{aljehane2015visualizing}, so data visualisation is also applied in higher education to allow international students to gain information on destination cities or universities straightforwardly and interactively. Therefore, it is a great value to combine data visualisation with open data to help international students make informed decisions for their higher education destination choices.



\section{Problem Statement
}

The impacts of economic and social events are likely to change the factors in the international students’ decision-making process for higher education. Some determining factors may become less important, while some less important factors may attract international students’ more attention. The result of Brexit Poll, for instance, is possible to motivate international students to focus on the location of destination universities or cities, because universities or cities near London can provide them with more opportunities to get work permits after graduation. Hence, it is necessary to conduct a survey to identify current determining factors that influence international students’ decision-making process to help them choose destination universities or cities in the UK. There are millions of open datasets in the world, so it is necessary to search for or collect relevant open datasets that reflect those factors influencing higher education destination choices. Moreover, with the advancement of visualisation techniques, it is also important to choose suitable techniques to visualise these open datasets to help students make informed decisions.



\section{Aims and Objectives 
}

The major aim of the project is to develop a web-based application that uses data visualisation techniques to visualise open data to help international students’ make informed decisions for their higher education destination universities or cities. The project also aims to research international students’ decision-making process for higher education and identify current significant factors in this process. The functionality and the user interface of the system will be aligned with those factors.

To accomplish the specified aims of this project, the objectives defined below identify the process that will be followed throughout the project.
\begin{itemize}


	\item Analyse and summarise current key factors that influence international students’ higher education destination choices in the UK.
 
	\item Search for or collect relevant open data that can present those significant factors affecting higher education destination choices.
  
	\item Investigate and identify advanced visualisation techniques that are suitable for visualising the important factors. 

	\item Apply appreciate data visualisation techniques in designing a web application that includes all significant factors and provides international students with comprehensive descriptions of destination universities or cities in the UK.

	\item Test and evaluate the web application to identify whether this application is helpful for international students to make informed decisions for their destination universities or cities.
\end{itemize}

\section{Dissertation Structure
}
The following chapters will emphasise on the procedures of analysing and developing a web-based application to visualise open data in higher education throughout the phases of this project. Specifically, the background research conducted related to the factors influencing higher education destination choices and the application of open data and data visualisation in educational choices is presented in chapter 2. Chapter 3 will introduce information on project management, such as project constraints, project methodology, time management as well as risk analysis and management. Subsequently, chapter 4 and chapter 5 will cover system analysis and system design respectively. Afterwards, chapter 6 will recount the implementation outcomes of this system based on system analysis and design in last two chapters, and introduce the highlighted features implemented in this system; Furthermore, the methodologies and procedures for testing and evaluation will be described in chapter 7. Eventually, the conclusion, critical reflection and future work will be given in chapter 8.

