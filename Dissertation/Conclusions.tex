%% ----------------------------------------------------------------
%% Conclusions.tex
%% ---------------------------------------------------------------- 
\chapter{Conclusion} \label{Chapter: Conclusion}
This chapter will firstly propose the future work that could improve this project based on the results of testing and evaluation. Afterwards, the critical reflection will reflect the learning outcomes and challenges by completing this project. Finally, the chapter will summarize this project based on the previous chapters.


\section{Future Work}

Due to the project constraints, the system can only offer functionalities that were initially identified and planned. Therefore, there are still some future work for this system. Firstly, a database can be added in this system to improve the stability and functionality. Although the usage of open data APIs can save time and obtain latest data, open APIs cannot be accessed without internet connection. The usage of a database allows to store the open datasets locally and ensures this system can be available at any time. Moreover, using a database make it possible to add more functionalities, such as login and registration. Secondly, open datasets about historical weather should be provided in this system, because some respondents suggested that information about historical weather is more helpful for them to compare the weather of different cities. Thirdly, the system should be improved to allow students to compare several universities and cities at the same time, which can improve the user experience. The usage of the database provides probabilities for this improvement. Lastly, more open datasets and data visualizations should be included in this system, especially in University\&Course and Ranking Table, and the web applications mentioned in section 2.3 are great examples.

\section{Critical Reflection}

It was a really challenge for me to go through the whole process of this project, because I have never completed a project independently before. Therefore, the completion of this project means the ending of a great journey of learning, which enriched my understanding on several areas and improved my practical skills. Firstly, background research on international students’ higher education destination choices, open data and data visualisation provided me with a great understanding of the relationship between them, which resulted in the final system of this project. Secondly, my project management understanding was improved throughout this project, because I had to choose a suitable software development methodology, make an appropriate plan and identify the potential risks to ensure this project can meet user requirements on time and within budget limits. Thirdly, the system design phase allows me to realise the importance of the stakeholders when developing a new idea, which make systems more than just a software tool. Fourthly, the implementation phase prompted me to learn some new technologies and tools to development a software system, though sometimes this phase was tough and dull. Lastly, I learned that the implementation phase is not the whole of the software development lifecycle, testing and evaluation are also crucial to the completion of a software system. 

Although I acquired a lot of knowledge and skill by experiencing each phase of this project, I also faced some challenges. As an international student, English is not my first language, so the first challenge for me was reading English literature and technical documentations. The second challenge was to search suitable open datasets and choose techniques and tools to implement this system. The third and biggest challenge was the implementation phase, because I did not learn JavaScript before and had to learn it from scratch and use it to develop this system within time limit.  However, the completion of this project means that I succeeded to complete those challenges.

To summarize, the aims and objectives of this project were successfully fulfilled, even though there is still some future work ahead of improving current functionalities and add more ones based on the results of comparative evaluation. By completion of the project, I have some ideas how to proceed a software project and have faith in becoming a software developer in future career.

\section{Conclusion}


International students’ decision-making process for higher education is complex and often involved in multiple factors. However, the factors influencing this process vary from different student populations and can be influenced by social and economic events. The dissertation summarised several key factors that influence international students’ destination choices from some literature and conducted a survey to identify those factors influencing the decision-making process of international students who choose the UK as their destination countries. Based on the results of the survey, this system was carefully analysed and designed to meet user requirements. Afterwards, a complete, fully-functional system was developed using open data and data visualization to help international students make informed decisions for their higher education destination choices.  Moreover, the testing and evaluation phase was followed to guarantee the high quality of the system and also to validate that all the intended functionalities are delivered. The results of the comparative evaluation showed some improvements should be considered in this system. For instance, it will be more helpful for international students if this system can provide information about historical weather instead of weather forecast. However, this system provides the ability to international students to compare different information about the key factors influencing their higher education destination choices. 

In conclusion, this system proved to be helpful for international students when they decide destination choices for their higher education in the UK. More importantly, this system provides a new way to apply open data and data visualization in higher education.

